%% $Id: libertinus-doc.tex 805 2018-09-04 07:22:01Z herbert $
\listfiles
\documentclass[english]{article}
\usepackage{libertinus}
\usepackage{babel}

\title{Type~1 or OpenType Libertinus font\\ --\\ Type~1 or OpenType}
\author{Herbert Voß}
\usepackage{parskip}
\parindent=0pt


\begin{document}
\maketitle


\begin{abstract}
The font family Libertinus is derived from the Linux Libertine and enhanced with a
math font.
\end{abstract}


\section{Meaning}

The package \texttt{libertinus} is only a wrapper for the other two
packages \texttt{libertinus-type1} and \texttt{libertinus-otf}. Depending
to the used \TeX-engine one of these packages is loaded. All optional
arguments are passed to \texttt{libertinus-type1} if \texttt{pdflatex} is used
or passed to \texttt{libertinus-otf} if \texttt{xelatex} or \texttt{lualatex}
are used.

\end{document}
